{
  \titleformat{\section}[block]{}{}{0pt}{\huge\textbf}
  \parskip=0.6em

\section{Kiel uzi la adreslibron?}

\subsection{Kien vi vojaĝas?}

Aŭ vi jam scias kien vi volas vojaĝi, aŭ vi inspiriĝas de Pasporta Servo
kaj planas vian vojaĝon legante la libron.

Vi legas la adresojn en la libreto, legas la kondiĉojn. Se vi trovas
adreson kiu taŭgas, vi kontaktas la gastiganton laŭ la petita formo kaj
tempo. Vi skribas retmesaĝon, telefonas, aŭ uzas la sistemon de la
retejo.

Strikte ne aperu sen kontakti la gastiganton! Tre-tre malmulte da
gastigantoj pretas tiel akcepti gastojn, kaj kutime ili mencias tion en
la kondiĉoj apud sia adreso.

La ora regulo estas: {\semibold ĉiam legu la kondiĉojn du foje}. Estas klare
skribite kiom frue vi devas peti tranokteblon (t), ĉu vi povas gasti kun
pluraj amikoj (g) devos kunporti dormsakon, kiom da noktoj vi povos
resti (n), ktp.

Vi devas atendi la konfirmon de la gastiganto, kaj eventuale priparoli
la detalojn de via alveno, manĝoj, aliaj petoj, kondiĉoj.

Se vi skribis al pluraj personoj en la sama urbo, vi helpas al la
gastigantoj se vi tion mencias. Se pro ajnaj ŝanĝoj en feriaj planoj vi
tamen ne povos veni al iu gastiganto, nepre sciigu pri tio la
gastiganto(j)n kiel eble plej frue.

\subsection{Kio okazas se iu ne povas akcepti vin?}

Tio kelkfoje okazas. Ju pli frue vi kontaktas homojn, ke vi deziras
tranokti des pli granda estas la sukceso. Gastiganto estas en la libro, ĉar
li/ŝi pretas gastigi. Do, kutime nea respondo estas se tute ne konvenas la
tempo.

\subsection{Ĉu vi eventuale povas manĝi ĉe la gastiganto?}

Jes, kompreneble. Sed kiel ĉio, ankaŭ tio dependas de interkonsento.
Gastiganto ne devas proponi manĝojn. Kelkaj klare skribas inter la
kondiĉoj, ke volonte ofertos manĝon al vi, aŭ ke vi mem aranĝu tion.

Se vi interkonsentas pri manĝado, sciu, ke la gastiganto rajtas peti
repagon de la kostoj por la manĝo. Vi kompreneble ankaŭ povas proponi
(prepari) manĝon por la gastiganto. Komuna kuirado estas agrable kaj
apud vespermanĝo oni povas bone babiladi kaj ekkoni unu la alian.

\subsection{La strukturo de la libreto}

La adresoj estas grupitaj laŭ landoj. Ili aperas en alfabeta sinsekvo.
Ene de la landoj la adresoj aperas laŭ regionoj, se estas multe da
gastigantoj. Alikaze la listigo okazas laŭ ``la plej proksimaj grandaj
urboj''.

Tiuj urboj ĉefe gravas, kiam temas pri malgranda vilaĝo, kiu eble
malfacile troveblas sur la mapo. La grupigo de la adresoj helpas vidi,
kiom da gastigantoj estas en ajna regiono.

La adreso komenciĝas per la nomo de la gastiganto (kiam naskiĝis),
kunloĝantoj (kun naskiĝdatoj), poste la poŝtkodo, la nomo de la urbo per
dikaj literoj, strato, telefonnumero(j), retadreso kaj fine la kondiĉoj.
La mallongigo (g) signifas, kiom da gastoj oni akceptas, la (n) por kiom
da noktoj, kaj (t) almenaŭ kiom da tagoj antaŭe vi kontaktu la
gastiganton. Fine estas aldonaj komentoj pri la tranoktado.

Se ajna parto de la adreso ``mankas'', tio signifas, ke la gastiganto
intence ne donis detalojn.

Post la listo de la adresoj troviĝas la mapoj. Ĉiu punkto montras
adreson de la libro. La indikitaj urboj sur la mapo estas de ``la plej
proksimaj urboj'', do reprezentas la grupigojn, ne la adresojn mem.

\subsubsection{Kontribuu al Pasporta Servo, helpante nin!}

Ni faras ĉion por aperigi nur ĝustajn informojn en la libro. Ni persone
kontaktis ĉiuj gastigantojn, kiuj troviĝas en la eldono.

Malgraŭ niaj streboj povas okazi, ke aperas neaktualaj gastigantoj,
alispecaj misoj, tajperaroj. Se vi trovas ion malĝustan aŭ plibonigeblan
ne hezitu kontakti nin!

Plej facile estas sendi la rimarkojn rekte al la kompilanto ĉe
saluton@pasportaservo.org

Vi povas helpi ankaŭ sendante mondonacon por la projekto tra UEA,
menciante ``por Pasporta Servo''.

Kaj kompreneble: iĝu gastiganto! Pasporta Servo vivas, ĉar multe da
aktivuloj pretas gastigi.

\subsection{Gravaj nocioj rilate al PS}

\begin{description}
\item[Landa Organizanto]
Varbas gastigantojn en sia lando kaj informas pri Pasporta Servo (foje
ankaŭ al ne-Esperantaj organizaĵoj), kaj helpas la kompilanton
diversmaniere. Se via adreso ŝanĝiĝas, kaj bezonas helpon por administri
tion, vi povas peti helpon de via Landa Organizanto. Dum la kompilado de
la libro ili zorgas pri la kontaktado de la gastigantoj kaj la ĝusteco
de la mapoj.

Kelkfoje Landa Organizanto organizas rondvojaĝojn aŭ donas informojn
(ekz. pri sia lando) al individuaj vojaĝemuloj. Kelkaj Landaj
Organizantoj ankaŭ vendas ekzemplerojn de Pasporta Servo, aŭ povas
informi pri (nacilingvaj) varbiloj pri Esperanto aŭ Pasporta Servo.
Adresojn de Landaj Organizantoj vi trovos komence ĉe la respektivaj
landoj en la adreslibro.

Multaj landoj ne havas Landan Organizanton. Dum la reviviga proceso de
la libro ni strebis trovi por ĉiu lando iun, sed ne sukcesis. Ni ĉiam
bezonas aktivulojn, kiuj plej bone konas la lokajn kondiĉojn,
Esperanto-parolantojn. Esti aktivulo por Pasporta Servo igas vin
ekkonatiĝi kun viaj samlandanoj. Se vi emas iĝi Landa Organizanto por
via lando ne hezitu kontakti nin ĉe saluton@pasportaservo.org!
\end{description}

\begin{description}
\item[Kompilanto]
Por la eldono de 2017 la kompilantoj de la libro estas Baptiste
Darthenay kiu okupiĝis pri la teĥnika parto de la libro, la programado
de la retejo kun la informoj de la gastigantoj kaj enpaĝigo de la libro.
Stela Besenyei-Merger okupiĝis pri la eksteraj rilatoj kun la
gastigantoj, sociaj retejoj kaj kontaktado, trejnado kaj helpado de la
Landaj Organizantoj.

Ajnan mesaĝon al la kompilantoj bonvolu sendi al:\\
\texttt{saluton@pasportaservo.org}
\end{description}

\begin{description}
\item[Loka Peranto]
Peras gastigojn en sia loko (urbo) kaj kontaktigas gastigantojn kun
vojaĝantoj. Li/ŝi eventuale eĉ konas adresojn de gastigantoj kiuj ne
aperas en la adreslibro.
\item[Centra Distribuanto]
UEA, la ĉefa vendanto kaj distribuanto de la adreslibro. UEA havas krome
ankaŭ stoketon da varbiloj por Pasporta Servo (kun aliĝilo).
\end{description}

Adreso aperas en la kolofono.

\begin{description}
\item[pasportaservo.org]
La oficiala retpaĝo de Pasporta Servo.

Uzantoj povas rapide ŝanĝi siajn datumojn, sen atendi la aperon de la
venonta adreslibro.

La paĝo estis plene refarita ekde la antaŭa versio el 2011. Momente ĝi
pli funkcias kiel reta datumbazo de gastigantoj kaj uzantoj. Ni uzos la
tempon nun, post la publikado de la libro por igi la retejon eĉ pli
facile uzebla kaj pli alloga al la publiko.
\end{description}

\subsection{Kiel ricevi la adreslibron?}

Esperantaj libroservoj kutime disponas pri stoko da adreslibroj, sed
ĉiam eblas mendi rekte de UEA (adreso en la kolofono). Krome vendas la
adreslibron kelkaj Landaj Organizantoj (vidu sub la koncerna lando), kaj
kelkaj Landaj Sekcioj de TEJO.

Gastigantoj en la adreslibro ne plu ricevas aŭtomate senpagan
ekzempleron pro la kostoj de produktado kaj sendado de la libro. Dankon,
se vi kiel gastiganto subtenas la projekton kaj aĉetis ekzempleron ankaŭ
por vi mem!
}
